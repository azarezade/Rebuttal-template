\documentclass[12pt]{article}
\usepackage{lipsum} % to generate some filler text


% import Eq and Section references from the main manuscript if needed
% \usepackage{xr}
% \externaldocument{manuscript}

% define counters for reviewers and their points
\newcounter{reviewer}
\setcounter{reviewer}{0}
\newcounter{point}[reviewer]
\setcounter{point}{0}

% This refines the format of how the reviewer/point reference will appear.
\renewcommand{\thepoint}{Q\,\thereviewer.\arabic{point}} 

% command declarations for reviewer points and our responses
\newcommand{\reviewersection}{\stepcounter{reviewer}
                  \section*{Reviewer \thereviewer}}
\newcommand{\point}[1]{\refstepcounter{point}  \bigskip \hrule \medskip \noindent 
               \textsl{{\fontseries{b}\selectfont \thepoint } \medskip #1}\par}
\newcommand{\reply}{\medskip \noindent \textbf{Reply}:\ }   


\begin{document}
% title of the paper
\begin{center}
\Large\bfseries
	Revision of: \\ 
	Title of Your Paper
\end{center}
% list of authors 
\begin{center}
\small
First Author, Second Author, and Last Author \\
\today
\end{center}


\title{Responses to Review Comments on ``Title of Your Paper'' \\
  \vspace{0.5em} \large \emph{First round of reviewing}}
\author{First Author \and Second Author \and Third Author}

\noindent
We thank the anonymous Reviewers for their efforts and feedback. 
The following briefly discusses how the paper was revised in response to the comments provided by the reviewers. In summary we have done this and that.


% General intro text goes here
\reviewersection

\point{Point one description \label{pt:foo}}
\reply{\lipsum[1]} 

\reviewersection

\point{First point of Reviewer \thereviewer}
\reply{This point was already answered as \ref{pt:foo}.}

\point{Description of second point \label{pt:bar}}
\reply{\lipsum[7-8]}

\end{document}